\documentclass[english]{sheftech}
\usepackage[latin1]{inputenc}
\usepackage{amsmath}
%\usepackage{caption}
%\usepackage{subfig}
\usepackage{graphicx}
\usepackage{amssymb}
\usepackage{epsfig}
\usepackage{natbib}

\makeatletter
%%%%%%%%%%%%%%%%%%%%%%%%%%%%%% Textclass specific LaTeX commands.
 
%%%%%%%%%%%%%%%%%%%%%%%%%%%%%% User specified LaTeX commands.

%\usepackage{ae,aecompl}

\usepackage{babel}
\makeatother
\begin{document}

\title{Supplementary material to ``A probabilistic dynamical model for 
quantitative inference of the regulatory mechanism of transcription''}
\author{Guido Sanguinetti & Magnus Rattray & Neil D. Lawrence}

\maketitle
In this document we discuss in depth some details of the computational model
which could not be discussed in the main paper due to space limitations. We 
also present more results in terms of TFA profiles and lists of active 
transcription factors.

\section{Dependence of the model on the choice of p-value in the ChIP data}

In the paper we chose to discretize the ChIP data by considering as
positive only bindings corresponding to a $p$-value smaller than
$10^{-3}$. This, as far as we are aware, is the standard procedure
followed by the authors of all other papers applying regression-based
methods to the problem of integrating microarray and ChIP data, so
it was an obvious choice in order to compare our results with other
methods.

The $p$-value cut-off was originally suggested in \cite{Lee02} as
providing a low level of false positives (estimated at 5\%) and an
acceptable level of false negatives (the authors estimated that approximately
a third of all bindings went undetected). It must be borne in mind,
though, that binding is only a necessary condition for regulation 
\citep{Martone03}.
Therefore, the fraction of false positives in the regulatory relationships
can be expected to be much higher.

To test the robustness of our model upon changes in the cut-off $p$-value,
we ran our model on the cell-cycle data set with three different connectivity
matrices: one was obtained by using the customary cut-off at $p=10^{-3}$,
another was obtained by taking $p=2\times10^{-4}$ and a third by
taking $p=5\times10^{-3}$. The data sets obtained were very different:
the most stringent $p$-value led to a smaller network with 1309 genes
and 88 transcription factors, while at the other end $p=5\times10^{-3}$
led to a large network with 3130 genes and 112 transcription factors.
All the results can be obtained by using the MATLAB code available
for download at http://umber.sbs.man.ac.uk/resources/puma.

We then constructed effective connectivity matrices by retaining only
regulatory relations that were predicted to be significant (at 95\%
significance level) by our model. The results of the analysis are
summarised in Table \ref{pVals}.

Not surprisingly, the fraction of significant regulations (out of
the possible total) is higher the more stringent the cut-off chosen.
However, while this fraction is very similar in the two most stringent
cases (23.0\% and 19.6\% respectively), it is smaller 
for $p=5\times10^{-3}$ (approximately
14.0\%). This suggests that the number of false positives at $p=5\times10^{-3}$
might be too large, forcing the model to explain behaviours that are
inconsistent and resulting in fewer confident predictions.

Comparing the results between the different experiments, we see that the
two more stringent $p$-values give similar results, with approximately
70\% of effective regulatory relations shared between the two predictions. 
This is somewhat surprising if we consider how different the two networks are. 
The run using $p=5\times10^{-3}$ gave less consistent results, with
only approximately 46\% of effective regulatory relations predicted
both at $p=10^{-3}$ and $p=5\times10^{-3}$. These relations were
almost all present also in the run using the most stringent $p$-value. 
These results changed only slightly if we altered the significance
threshold for the effective connectivity matrix.

This analysis indicates that, consistently with the suggestions of
\cite{Lee02}, $p=10^{-3}$ provides a good choice of a cut-off to
discretize ChIP data, as it seems to capture a large enough number
of regulatory relationships while at the same time keeping the number
of false positives at a reasonable level. 
\begin{table}
\small\center\begin{tabular}{|c|c|c|}
\hline 
$p$-value&
possible links&
significant links\tabularnewline
\hline

\hline 
2$\times10^{-4}$&
2111&
486\tabularnewline
\hline 
1$\times10^{-3}$&
3656&
716\tabularnewline
\hline
5$\times10^{-3}$&
7200&
1007\tabularnewline

\hline 
\end{tabular}
\vspace{0.2cm}

\caption\small{Possible regulations and significant regulations for three different 
choices of cut-off in the $p$-value for ChIP data \label{pVals}}
%\vspace{-0.7cm}
\end{table}



\section{Global analysis of regulatory networks}

We then considered the global aspects of the inferred regulations
in both the cell cycle case and metabolic cycle case. To assess
the significance of a relationship we considered the ratio between the changes
across time of the gene-specific TFA and the associated standard deviation,
considering ratios greater than 2 to be significant at 95\% confidence level. 
The results of this analysis are summed up in Table \ref{global}.

\begin{table}
\small\center\begin{tabular}{|c|c|c|c|}
\hline 
Data set&
No of genes&
No of TFs&
genes with multiple regulators\tabularnewline
\hline

\hline 
Cell cycle&
522&
47&
23\%\tabularnewline
\hline 
Metabolic cycle&
2167&
151&
11\%\tabularnewline


\hline 
\end{tabular}
\vspace{0.2cm}

\caption\small{Global properties of the networks of significant regulations 
for the two data sets studied.\label{global}}
%\vspace{-0.7cm}
\end{table}
In the cell cycle data set (with connectivity obtained using a $p$-value
of $10^{-3}$), the model inferred 716 significant regulatory relations.
These were due to 47 transcription factors acting on 522 genes. Out
of the 47 transcription factors, 27 are confirmed transcription factors
active during the cell cycle \cite{Luscombe04}. These account for 466
regulatory relations. Of the 522 genes involved, 119 had
more than one significant regulator. A list of the transcription factors
involved, together with the number of genes they significantly regulate
and a comparison with the data from \cite{Luscombe04} is included
in the attached spreadsheet CellCycleTF.xls.

In the metabolic cycle data, our model detected 2410 significant regulations
involving 151 transcription factors and 2167 genes. Notice that in this data set
the fraction of significant regulations out of the total possible
is much higher than in the cell cycle at the same level of
significance (approximately 42\% versus 20\%). This is probably due to the fact
that, in the metabolic cycle data set, we were able to use the noise
information extracted at probe level using the mmgMOS algorithm 
\cite{Liu:mmgmos05}, resulting in a more principled treatment of the noise. 
In the metabolic
cycle, 236 genes appear to have multiple significant regulators, 
five of which were regulated
by three transcription factors and one by four.
A list of the transcription factors
involved and the number of genes each of them regulates is included
in the attached spreadsheet MetabolCycleTF.xls.
\section{Further TFAs}
In the main paper there was only space to show the inferred TFAs only in very few cases (ACE2 
for the cell cycle data set, LEU3 and ACE2 for the metabolic cycle data set). 
We show here more gene-specific TFAs and compare them with non-specific 
profiles obtained from regression. Further examples can be obtained by using
the online MATLAB code.

Figure \ref{cellCycle} shows the TFAs of four more transcription factors that
are involved in the cell cycle according to our results. The TFA obtained by regression for these transcription factors is 
shown in the first column, while the other columns show the gene-specific
TFAs obtained with our model for the three most significantly regulated
targets of these transcription factors. Notice that some gene-specific TFAs 
look quite different from the TFAs obtained by regression. For example, the TFA
obtained by regression for STE12 seem to be dominated by white noise, while the
gene specific TFA on the most significant targets shows a very different 
behaviour, being stationary for the first part of the cycle and peaking 
towards the end. The gene-specific TFAs of MBP1's three main targets again 
show different behaviours among them, and in turn different from the regression
picture. 

Figure \ref{metCycle} shows the TFAs of four more transcription factors that
significantly regulate five or more genes in the metabolic cycle but do not 
have periodic expression according to \cite{Tu05}. These are ABF1, ARO80, FHL1 
and SMP1. The TFA obtained by regression for these transcription factors is 
shown in the first column, while the other columns show the gene-specific
TFAs obtained with our model for the three most significantly regulated
targets of these transcription factors. These results indicate that, even if
the expression levels of these transcription factors is not periodic, their
gene-specific activities are to be considered periodic for many genes. 
We would suggest that the appropriate criterion to determine whether a 
transcription factor is involved in a periodic cellular process is whether its 
activities, rather than its expression level, display periodic behaviour.
\begin{figure}
\vspace{-1cm}
%\begin{figure}%\vspace{-1cm}
\setlength{\unitlength}{0.8cm} \begin{center}
    \begin{picture}(18,17) \epsfysize = 3cm
\epsfxsize=3cm
    \put(-1.8,0){
\epsfbox{/home/guido/mlprojects/chipChip/tex/figures/SMP1Regression.eps}}
      \put(1.4,-0.2){\mbox{$t$}}
    \put(-2.3,3.5){\mbox{\tiny{$TFA$}}}\put(0,-0.5){\mbox{(m)}}
\epsfxsize=3cm        \epsfysize =3cm \put(2.9,0){
\epsfbox{/home/guido/mlprojects/chipChip/tex/figures/SMP1SLT2.eps}}
           \put(2.4 ,3.5){\mbox{\tiny{$TFA$}}}
        \put(5.9,-0.2){\mbox{$t$}}\put(4.7,-0.5){\mbox{(n)}}
\epsfxsize=3cm        \epsfysize =3cm \put(7.7,0){
\epsfbox{/home/guido/mlprojects/chipChip/tex/figures/SMP1GRX5.eps}}
   \put(7.3 ,3.5){\mbox{\tiny{$TFA$}}}\put(11,-0.2){\mbox{$t$}}
\put(9.5,-0.5){\mbox{(o)}}
\epsfxsize=3cm        \epsfysize =3cm \put(12.3,0){
\epsfbox{/home/guido/mlprojects/chipChip/tex/figures/SMP1HKR1.eps}}
   \put(11.9 ,3.5){\mbox{\tiny{$TFA$}}}\put(15.8,-0.2){\mbox{$t$}}
\put(14.2,-0.5){\mbox{(p)}}
\epsfysize = 3cm
\epsfxsize=3cm
    \put(-1.8,4.7){
\epsfbox{/home/guido/mlprojects/chipChip/tex/figures/FHL1Regression.eps}}
      \put(1.4,4.5){\mbox{$t$}}
    \put(-2.3,8.2){\mbox{\tiny{$TFA$}}}\put(0,4.2){\mbox{(i)}}
\epsfxsize=3cm        \epsfysize =3cm \put(2.9,4.7){
\epsfbox{/home/guido/mlprojects/chipChip/tex/figures/FHL1RPL9A.eps}}
           \put(2.4 ,8.2){\mbox{\tiny{$TFA$}}}
        \put(5.9,4.5){\mbox{$t$}}\put(4.7,4.2){\mbox{(j)}}
\epsfxsize=3cm        \epsfysize =3cm \put(7.7,4.7){
\epsfbox{/home/guido/mlprojects/chipChip/tex/figures/FHL1RPL14B.eps}}
   \put(7.3 ,8.2){\mbox{\tiny{$TFA$}}}
\put(11,4.5){\mbox{$t$}}\put(9.5,4.2){\mbox{(k)}}
\epsfxsize=3cm        \epsfysize =3cm \put(12.3,4.7){
\epsfbox{/home/guido/mlprojects/chipChip/tex/figures/FHL1RPL9B.eps}}
   \put(11.9 ,8.2){\mbox{\tiny{$TFA$}}}\put(15.8,4.5){\mbox{$t$}}
\put(14.2,4.2){\mbox{(l)}}
\epsfysize = 3cm
\epsfxsize=3cm
    \put(-1.8,9.4){
\epsfbox{/home/guido/mlprojects/chipChip/tex/figures/ARO80Regression.eps}}
      \put(1.4,9.2){\mbox{$t$}}
    \put(-2.3,12.9){\mbox{\tiny{$TFA$}}}\put(0,8.9){\mbox{(e)}}
\epsfxsize=3cm        \epsfysize =3cm \put(2.9,9.4){
\epsfbox{/home/guido/mlprojects/chipChip/tex/figures/ARO80YNL124W.eps}}
           \put(2.4 ,12.9){\mbox{\tiny{$TFA$}}}
        \put(5.9,9.2){\mbox{$t$}}\put(4.7,8.9){\mbox{(f)}}
\epsfxsize=3cm        \epsfysize =3cm \put(7.7,9.4){
\epsfbox{/home/guido/mlprojects/chipChip/tex/figures/ARO80ARH1.eps}}
   \put(7.3 ,12.9){\mbox{\tiny{$TFA$}}}
\put(11,9.2){\mbox{$t$}}\put(9.5,8.9){\mbox{(g)}}
\epsfxsize=3cm        \epsfysize =3cm \put(12.3,9.4){
\epsfbox{/home/guido/mlprojects/chipChip/tex/figures/ARO80ARO10.eps}}
   \put(11.9,12.9){\mbox{\tiny{$TFA$}}}\put(15.8,9.2){\mbox{$t$}}
\put(14.2,8.9){\mbox{(h)}}
\epsfysize = 3cm
\epsfxsize=3cm
    \put(-1.8,13.9){
\epsfbox{/home/guido/mlprojects/chipChip/tex/figures/ABF1Regression.eps}}
      \put(1.4,13.7){\mbox{$t$}}
    \put(-2.3,17.4){\mbox{\tiny{$TFA$}}}\put(0,13.4){\mbox{(a)}}
\epsfxsize=3cm        \epsfysize =3cm \put(2.9,13.9){
\epsfbox{/home/guido/mlprojects/chipChip/tex/figures/ABF1YOR309C.eps}}
           \put(2.4 ,17.4){\mbox{\tiny{$TFA$}}}
        \put(5.9,13.7){\mbox{$t$}}\put(4.7,13.4){\mbox{(b)}}
\epsfxsize=3cm        \epsfysize =3cm \put(7.7,13.9){
\epsfbox{/home/guido/mlprojects/chipChip/tex/figures/ABF1RFA2.eps}}
   \put(7.3 ,17.4){\mbox{\tiny{$TFA$}}}
\put(11,13.7){\mbox{$t$}}\put(9.5,13.4){\mbox{(c)}}
\epsfxsize=3cm        \epsfysize =3cm \put(12.3,13.9){
\epsfbox{/home/guido/mlprojects/chipChip/tex/figures/ABF1YPL012W.eps}}
   \put(11.9 ,17.4){\mbox{\tiny{$TFA$}}}\put(15.8,13.7){\mbox{$t$}}
\put(14.2,13.4){\mbox{(d)}}
\end{picture}\end{center}
\caption{TFAs and gene-specific TFAs for some transcription factors active in
the yeast metabolic cycle which have non periodic expression levels. (a) TFA 
of ABF1 obtained by regression. (b-d) TFA of ABF1 for its three main targets,
YOR309C, RFA2 and YPL012W respectively. (e) TFA of ARO80 obtained by 
regression. (f-h) TFA of ARO80 for its three main targets, YNL124W, ARH1
and ARO10 respectively. (i) TFA of FHL1 obtained by regression. (j-l) TFA
of FHL1 for its three main targets, RPL9A, RPL14B and RPL9B respectively.
(m) TFA of SMP1 obtained by regression. (n-p) TFA of SMP1 for its three main
targets, SLT2, GRX5 and HKR1 respectively.}\label{metCycle} 
\end{figure}
\begin{figure}
%\vspace{-1cm}
%\begin{figure}%\vspace{-1cm}
\setlength{\unitlength}{0.8cm} \begin{center}
    \begin{picture}(18,18) \epsfysize = 3cm
\epsfxsize=3cm
    \put(-1.8,0){
\epsfbox{/home/guido/mlprojects/chipChip/tex/figures/NDD1Regression.eps}}
      \put(1.4,-0.2){\mbox{$t$}}
    \put(-2.3,3.5){\mbox{\tiny{$TFA$}}}\put(0,-0.5){\mbox{(m)}}
\epsfxsize=3cm        \epsfysize =3cm \put(2.9,0){
\epsfbox{/home/guido/mlprojects/chipChip/tex/figures/NDD1PHO3.eps}}
           \put(2.4 ,3.5){\mbox{\tiny{$TFA$}}}
        \put(5.9,-0.2){\mbox{$t$}}\put(4.7,-0.5){\mbox{(n)}}
\epsfxsize=3cm        \epsfysize =3cm \put(7.7,0){
\epsfbox{/home/guido/mlprojects/chipChip/tex/figures/NDD1YDR033W.eps}}
   \put(7.3 ,3.5){\mbox{\tiny{$TFA$}}}\put(11,-0.2){\mbox{$t$}}
\put(9.5,-0.5){\mbox{(o)}}
\epsfxsize=3cm        \epsfysize =3cm \put(12.7,0){
\epsfbox{/home/guido/mlprojects/chipChip/tex/figures/NDD1NCE102.eps}}
   \put(12.3 ,3.5){\mbox{\tiny{$TFA$}}}\put(15.5,-0.2){\mbox{$t$}}
\put(14.5,-0.5){\mbox{(p)}}
\epsfysize = 3cm
\epsfxsize=3cm
    \put(-1.8,4.7){
\epsfbox{/home/guido/mlprojects/chipChip/tex/figures/SWI5Regression.eps}}
      \put(1.4,4.5){\mbox{$t$}}
    \put(-2.3,8.2){\mbox{\tiny{$TFA$}}}\put(0,4.2){\mbox{(i)}}
\epsfxsize=3cm        \epsfysize =3cm \put(2.9,4.7){
\epsfbox{/home/guido/mlprojects/chipChip/tex/figures/SWI5PIR1.eps}}
           \put(2.4 ,8.2){\mbox{\tiny{$TFA$}}}
        \put(5.9,4.5){\mbox{$t$}}\put(4.7,4.2){\mbox{(j)}}
\epsfxsize=3cm        \epsfysize =3cm \put(7.7,4.7){
\epsfbox{/home/guido/mlprojects/chipChip/tex/figures/SWI5PIR3.eps}}
   \put(7.3 ,8.2){\mbox{\tiny{$TFA$}}}
\put(11,4.5){\mbox{$t$}}\put(9.5,4.2){\mbox{(k)}}
\epsfxsize=3cm        \epsfysize =3cm \put(12.7,4.7){
\epsfbox{/home/guido/mlprojects/chipChip/tex/figures/SWI5ASH1.eps}}
   \put(12.3 ,8.2){\mbox{\tiny{$TFA$}}}\put(15.5,4.5){\mbox{$t$}}
\put(14.5,4.2){\mbox{(l)}}
\epsfysize = 3cm
\epsfxsize=3cm
    \put(-1.8,9.4){
\epsfbox{/home/guido/mlprojects/chipChip/tex/figures/STE12Regression.eps}}
      \put(1.4,9.2){\mbox{$t$}}
    \put(-2.3,12.9){\mbox{\tiny{$TFA$}}}\put(0,8.9){\mbox{(e)}}
\epsfxsize=3cm        \epsfysize =3cm \put(2.9,9.4){
\epsfbox{/home/guido/mlprojects/chipChip/tex/figures/STE12FUS1.eps}}
           \put(2.4 ,12.9){\mbox{\tiny{$TFA$}}}
        \put(5.9,9.2){\mbox{$t$}}\put(4.7,8.9){\mbox{(f)}}
\epsfxsize=3cm        \epsfysize =3cm \put(7.7,9.4){
\epsfbox{/home/guido/mlprojects/chipChip/tex/figures/STE12KAR4.eps}}
   \put(7.3 ,12.9){\mbox{\tiny{$TFA$}}}
\put(11,9.2){\mbox{$t$}}\put(9.5,8.9){\mbox{(g)}}
\epsfxsize=3cm        \epsfysize =3cm \put(12.7,9.4){
\epsfbox{/home/guido/mlprojects/chipChip/tex/figures/STE12SST2.eps}}
   \put(12.3 ,12.9){\mbox{\tiny{$TFA$}}}\put(15.5,9.2){\mbox{$t$}}
\put(14.5,8.9){\mbox{(h)}}
\epsfysize = 3cm
\epsfxsize=3cm
    \put(-1.8,13.9){
\epsfbox{/home/guido/mlprojects/chipChip/tex/figures/MBP1Regression.eps}}
      \put(1.4,13.7){\mbox{$t$}}
    \put(-2.3,17.4){\mbox{\tiny{$TFA$}}}\put(0,13.4){\mbox{(a)}}
\epsfxsize=3cm        \epsfysize =3cm \put(2.9,13.9){
\epsfbox{/home/guido/mlprojects/chipChip/tex/figures/MBP1MRP8.eps}}
           \put(2.4 ,17.4){\mbox{\tiny{$TFA$}}}
        \put(5.9,13.7){\mbox{$t$}}\put(4.7,13.4){\mbox{(b)}}
\epsfxsize=3cm        \epsfysize =3cm \put(7.7,13.9){
\epsfbox{/home/guido/mlprojects/chipChip/tex/figures/MBP1AGA1.eps}}
   \put(7.3 ,17.4){\mbox{\tiny{$TFA$}}}
\put(11,13.7){\mbox{$t$}}\put(9.5,13.4){\mbox{(c)}}
\epsfxsize=3cm        \epsfysize =3cm \put(12.7,13.9){
\epsfbox{/home/guido/mlprojects/chipChip/tex/figures/MBP1YMR215W.eps}}
   \put(12.3 ,17.4){\mbox{\tiny{$TFA$}}}\put(15.5,13.7){\mbox{$t$}}
\put(14.5,13.4){\mbox{(d)}}
\end{picture}\end{center}
\caption{TFAs and gene-specific TFAs for some transcription factors active in
the yeast cell cycle. (a) TFA 
of NDD1 obtained by regression. (b-d) TFA of NDD1 for its three main targets,
PHO3, YDR033W and NCE102 respectively. (e) TFA of SWI5 obtained by 
regression. (f-h) TFA of SWI5 for its three main targets, PIR1, PIR3 
and ASH1 respectively. (i) TFA of STE12 obtained by regression. (j-l) TFA
of STE12 for its three main targets, FUS1, KAR4 and SST2 respectively.
(m) TFA of MBP1 obtained by regression. (n-p) TFA of MBP1 for its three main
targets, MRP8, AGA1 and YMR215W respectively.}\label{cellCycle} 
\end{figure}
\bibliographystyle{plain}
\bibliography{chip}

\end{document}
